\documentclass[bachelor, och, coursework]{SCWorks}
% параметр - тип обучения - одно из значений:
%    spec     - специальность
%    bachelor - бакалавриат (по умолчанию)
%    master   - магистратура
% параметр - форма обучения - одно из значений:
%    och   - очное (по умолчанию)
%    zaoch - заочное
% параметр - тип работы - одно из значений:
%    referat    - реферат
%    coursework - курсовая работа (по умолчанию)
%    diploma    - дипломная работа
%    pract      - отчет по практике
% параметр - включение шрифта
%    times    - включение шрифта Times New Roman (если установлен)
%               по умолчанию выключен
\usepackage{subfigure}
\usepackage{tikz,pgfplots}
\pgfplotsset{compat=1.5}
\usepackage{float}

%\usepackage{titlesec}
\setcounter{secnumdepth}{4}
%\titleformat{\paragraph}
%{\normalfont\normalsize}{\theparagraph}{1em}{}
%\titlespacing*{\paragraph}
%{35.5pt}{3.25ex plus 1ex minus .2ex}{1.5ex plus .2ex}

\titleformat{\paragraph}[block]
{\hspace{1.25cm}\normalfont}
{\theparagraph}{1ex}{}
\titlespacing{\paragraph}
{0cm}{2ex plus 1ex minus .2ex}{.4ex plus.2ex}

% --------------------------------------------------------------------------%


\usepackage[T2A]{fontenc}
\usepackage[utf8]{inputenc}
\usepackage{graphicx}
\graphicspath{ {./images/} }
\usepackage{tempora}

\usepackage[sort,compress]{cite}
\usepackage{amsmath}
\usepackage{amssymb}
\usepackage{amsthm}
\usepackage{fancyvrb}
\usepackage{listings}
\usepackage{listingsutf8}
\usepackage{longtable}
\usepackage{array}
\usepackage[english,russian]{babel}

% \usepackage[colorlinks=true]{hyperref}
\usepackage{url}

\usepackage{underscore}
\usepackage{setspace}
\usepackage{indentfirst} 
\usepackage{mathtools}
\usepackage{amsfonts}
\usepackage{enumitem}
\usepackage{tikz}
\usepackage{minted}

\newcommand{\eqdef}{\stackrel {\rm def}{=}}
\newcommand{\specialcell}[2][c]{%
\begin{tabular}[#1]{@{}c@{}}#2\end{tabular}}

\renewcommand\theFancyVerbLine{\small\arabic{FancyVerbLine}}

\newtheorem{lem}{Лемма}

\begin{document}

% Кафедра (в родительном падеже)
\chair{теоретических основ компьютерной безопасности и криптографии}

% Тема работы
\title{Обнаружение сетевого RDP трафика методом анализа его поведения}

% Курс
\course{3}

% Группа
\group{331}

% Факультет (в родительном падеже) (по умолчанию "факультета КНиИТ")
\department{факультета КНиИТ}

% Специальность/направление код - наименование
%\napravlenie{09.03.04 "--- Программная инженерия}
%\napravlenie{010500 "--- Математическое обеспечение и администрирование информационных систем}
%\napravlenie{230100 "--- Информатика и вычислительная техника}
%\napravlenie{231000 "--- Программная инженерия}
\napravlenie{10.05.01 "--- Компьютерная безопасность}

% Для студентки. Для работы студента следующая команда не нужна.
% \studenttitle{Студентки}

% Фамилия, имя, отчество в родительном падеже
\author{Токарева Никиты Сергеевича}

% Заведующий кафедрой
\chtitle{} % степень, звание
\chname{Абросимов М. Б.}

%Научный руководитель (для реферата преподаватель проверяющий работу)
\satitle{доцент} %должность, степень, звание
\saname{Гортинский А. В.}

% Руководитель практики от организации (только для практики,
% для остальных типов работ не используется)
% \patitle{к.ф.-м.н.}
% \paname{С.~В.~Миронов}

% Семестр (только для практики, для остальных
% типов работ не используется)
%\term{8}

% Наименование практики (только для практики, для остальных
% типов работ не используется)
%\practtype{преддипломная}

% Продолжительность практики (количество недель) (только для практики,
% для остальных типов работ не используется)
%\duration{4}

% Даты начала и окончания практики (только для практики, для остальных
% типов работ не используется)
%\practStart{30.04.2019}
%\practFinish{27.05.2019}

% Год выполнения отчета
\date{2022}

\maketitle

% Включение нумерации рисунков, формул и таблиц по разделам
% (по умолчанию - нумерация сквозная)
% (допускается оба вида нумерации)
% \secNumbering

%-------------------------------------------------------------------------------------------

\tableofcontents

\intro
Информация -- это сведения об окружающем мире и протекающих в нём процессах, которые зафиксированы на каком-либо носителе.
Благодаря протоколам удаленного доступа можно распоряжаться базами данных, информацией, которая хранится на другом устройстве. В недавнем прошлом большинство
схем удаленного доступа характеризовалось высокой стоимостью, низкой производительностью, небольшой скоростью передачи данных, недостаточным уровнем защищенности
передаваемой информации \cite{1}. 

Сейчас, когда практически все предприятия перешли на дистанционный формат работы, компании выбирают протокол RDP, так как он прост в настройке и в использовании.
Но далеко не все уделяют особое внимание безопасности собственных рабочих мест. Поэтому предприятия могут быть атакованы злоумышленниками.

В данной работе будут разобраны принцип работы RDP, анализ его поведения, а также методы обнаружения.

\section{Определение RDP}

Протокол RDP (от англ. Remote Desktop Protocol --- протокол удалённого рабочего стола) --- патентованный протокол 
прикладного уровня компании Microsoft и приобретен ею у другой компании Polycom, который предоставляет пользователю графический интерфейс для 
подключения к другому компьютеру через сетевое соединение. Для этого пользователь запускает клиентское программное обеспечение RDP, а на другом 
компьютере должно быть запущено программное обеспечение сервера RDP \cite{2}.

Клиенты для подключения по RDP существуют для большинства версий Microsoft Windows, Linux, Unix, macOS, iOS, Android и 
других операционных систем. Стоит отметить, что RDP-серверы встроены в операционные системы Windows. По умолчанию подключения, созданные с 
помощью RDP, используют UDP-порт 3389 и TCP порт 3389, по которым осуществляется передача данных.

\subsection{Безопасность протокола RDP}

Как уже известно, что для операционной системы Windows постоянно выходят различные обновления, включая обновлений RDS (от англ. Remote Desktop Services --- службы 
удаленных рабочих столов). В связи с этим возникают различные уязвимости при инициализации RDP-сессии. В основном они не связаны непосредственно с
протоколом RDP, но касаются службы удаленных рабочих столов RDS и позволяют при успешной эксплуатации путем отправления специального запроса через RDP
получить возможность выполнения произвольного кода на уязвимой системе, даже не проходя при этом процедуру проверки подлинности. Достаточно лишь иметь доступ
к хосту или серверу с уязвимой системой Windows. Таким образом, любая система, доступная из сети Интернет, является уязвимой при отсутствии установленных
последних обновлений безопасности Windows.

Если стоит задача защитить удаленный доступ, то, конечно, необходимо использовать надежный пароль, обновить свое программное обеспечение до последней версии,
также можно использовать VPN подключение, чтобы получить IP-адрес виртуальной сети и добавить его в правило исключения брандмауэра RDP. Стоит отметить, что
существует много разных способов, чтобы защитить подключение с помощью протокола RDP и более подробно это описано в документации Microsoft.

  \section{Принцип работы протокола RDP и анализ его поведения}

  Принцип работы RDP базируется на протоколе TCP. Соединение клиент-сервер происходит на транспортном уровне. После инициализации пользователь 
  проходит аутентификацию. В случае успешного подтверждения сервер передает клиенту управление. Стоит отметить, что под понятием слова <<клиент>> подразумевается
  любое устройство (персональный компьютер, планшет или смартфон), а <<сервер>> --- удаленный компьютер, к которому оно подключается.

  Протокол RDP внутри себя поддерживает виртуальные каналы, через которые пользователю передаются дополнительные функции операционной системы,
  например, можно распечатать документ, воспроизвести видео или скопировать файл в буфер обмена.

  % Известно, что RDP является прикладным протоколом, базирующимся на TCP. Для начала пользователю необходимо установить соединение клиент-сервер, которое
  % происходит на транспортном уровне. После инициализации RDP-сессии производится аутентификация. Далее сервер начинает передавать клиенту графический вывод и
  % ожидает входные данные от клавиатуры и мыши. В качестве графического вывода может выступать как точная копия графического экрана, передаваемая как изображение,
  % так и команды на отрисовку графических примитивов, например, линия, круг, эллипс, текст и др. Для протокола RDP приоритетом является передача вывода с помощью
  % примитивов, так как это экономит трафик. Изображение передается только в том случае, если не удалось согласовать параметры передачи примитивов при установке 
  % RDP-сессии. Обработка полученных команд и вывод изображения осуществляется с помощью графической подсистемы RDP-клинта. Сигнал нажатия и отпускания клавиши клавиатуры
  % шифруются и ожидают команды отправки \cite{3}.
  
  Далее в работе будет описан процесс установки RDP-сессии, во время которой осуществляется захват трафика с помощью одной известной программы Wireshark. С помощью нее
  можно достаточно подробно рассмотреть структуру сообщений протоколов.

  Для начала будет произведено подключение с помощью <<Удаленного рабочего стола>>. Это средство представляет собой встроенную в Windows программу, предназначенную
  для удалённого доступа. В качестве клиента и сервера будут выступать компьютеры с операционной системой Windows 10 Professional версии 21H2. Стоит отметить, что
  программа <<Удаленный рабочий стол>> может работать только в том случае, если клиент имеет операционную систему Windows, macOS, Android и iOS и 
  сервер может находится на платформах Windows, сделанных только в редакциях Professional, Enterprise и Ultimate. Поэтому, пользуясь данной программой, 
  не к каждой платформе можно будет подключится.
  
  Для подключения к удаленному рабочему столу были заданы статические IP-адреса. Клиенту был присвоен статический IP-адрес 192.168.10.254,
  а серверу --- 192.168.10.229, соответственно маска сети 255.255.255.0. После того, как были заданы IP-адреса, необходимо зайти в настройки
  Windows, чтобы включить возможность подключения к удаленному рабочему столу. Об этом более подробно описано
  в статьях \cite{userdp1} и \cite{userdp2}. Далее на сервере был произведен запуск анализа трафика с помощью приложения Wireshark.
  После подключения к удаленному компьютеру программа-анализатор трафика начала <<захватывать>> пакеты, как показано на рисунке \ref{wireshark1},
  принадлежащие следующим протоколам:
  
  \begin{figure}[H]
    \centering
    \includegraphics[width=0.8\textwidth]{photo/wireshark1.png}
    \caption{Окно программы Wireshark после захвата трафика}
    \label{wireshark1}
  \end{figure}

  \begin{itemize}
    \item RDPUDP --- протокол RDP, использующий для передачи данных UDP-протокол.
    \item RDPUDP2 также относится к протоколу RDP. Он был разработан для повышения производительности сетевого соединения по сравнению
    с соответствующим соединением RDP-UDP \cite{rdpudp}. 
    \item TLSv1.2 --- протокол защиты транспортного уровня, обеспечивающий защищенную передачу между узлами в сети интернет. В данном случае обеспечивает
    безопасность RDP-сессии.
  \end{itemize}
  
  Во время работы программы Wireshark было найдено достаточное количество пакетов, принадлежащих RDP, которые содержат в себе достаточно интересную
  информацию. Поэтому стоит рассказать о том, как происходит стандартный способ защиты RDP. Это можно представить в несколько этапов:

  \begin{enumerate}
    \item Клиент объявляет серверу о своем намерении использовать стандартный протокол RDP.
    \item Сервер соглашается с этим и отправляет клиенту свой собственный открытый ключ, полученный при шифровании алгоритмом RSA, а также некоторую строку
    случайных байтов (обычно её называют <<random сервером>>), генерируемую сервером. На рисунке \ref{rndserv} можно увидеть запись random сервера.

    \begin{figure}[H]
      \centering
      \includegraphics[width=0.9\textwidth]{photo/rndserv.png}
      \caption{Содержимое пакета, посылаемого от сервера клиенту (запись random сервера)}
      \label{rndserv}
    \end{figure}

    Совокупность открытого ключа и некоторая строка случайных байтов называется <<сертификатом>>. Данная запись изображена на рисунке \ref{cert}.
    
    \begin{figure}[H]
      \centering
      \includegraphics[width=0.9\textwidth]{photo/cert.png}
      \caption{Содержимое пакета, посылаемого от сервера клиенту (запись сертификата)}
      \label{cert}
    \end{figure}
    
    Сертификат подписывается службой терминалов, например, RDS, с использованием закрытого ключа для обеспечения подлинности.

    \item Теперь клиент посылает некоторую строку случайных байтов, которая называется <<premaster secret>>, показанная на рисунке \ref{cert1}. 
    
    \begin{figure}[H]
      \centering
      \includegraphics[width=0.9\textwidth]{photo/cert1.png}
      \caption{Содержимое пакета, посылаемого от клиента серверу (запись premaster secret)}
      \label{cert1}
    \end{figure}
    
    Данная запись шифруется открытым ключом, которая может быть расшифрована сервером только с помощью закрытого ключа службы терминалов.
    \item Сервер расшифровывает premaster secret с помощью собственного закрытого ключа.
    \item В случае успеха клиент и сервер получают свои сеансовые ключи из random сервера и premaster secret. Далее они используются для симметричного 
    шифрования остальной части сеанса.
  \end{enumerate}

  Теперь после того, как был произведен разбор RDP-сессии можно перейти к её обнаружению.

  
  \section{Обнаружение сеанса удаленного управления}

  Помимо программы <<Удаленный рабочий стол>> есть и другие приложения, с помощью которых можно установить соединение клиент-сервер. Например:
  
  \begin{enumerate}
    \item Удаленный рабочий стол Chrome (Chrome Remote Desktop) --- удаленный рабочий стол Chrome позволяет пользователям получать удаленный
    доступ к другому компьютеру через браузер Chrome. С помощью данного приложения можно подключаться к платформам, на которых есть этот браузер.
    Однако подключение таким образом к телефону невозможно, так как мобильное приложение «Удалённый рабочий стол Chrome» предоставляет доступ к компьютеру.
    \item TeamViewer --- приложение, позволяющее установить соединение с любым персональным компьютером или сервером всего за несколько секунд. На данный
    момент это очень популярное приложение, которое позволяет записывать сеансы на видео, общаться участникам в голосовом и текстовом каналах и открывать
    удалённый доступ только к выбранным приложениям.
    \item Remote Utilities --- программа для удалённого подключения к компьютерам. Серверная часть Remote Utilities устанавливается только на Windows,
    зато клиенты доступны на всех популярных платформах.
    \item Ammyy Admin --- надежный и удобный инструмент для удаленного доступа к компьютеру. Программа позволяет удалённо перезагружать компьютер,
    входить в систему и менять пользователей. Однако она доступна только для Windows.
  \end{enumerate}

  Сейчас таких программ, позволяющих подключиться к удаленному рабочему столу, стало достаточно много. И в некоторых приложениях уже не используется RDP.
  Например, программа Chrome Remote Desktop работает с протоколом HTTP, а TeamViewer вообще использует собственный проприетарный протокол, который не
  задокументирован. Хотя он немного похож на RDP по назначению, но включает в себя обход преобразования сетевых адресов и имеет немного другие методы аутентификации.
  Поэтому некоторые программы, созданные для распознавания и перехвата RDP трафика, в данном случае будут бесполезны.

  Хорошо, что такие программы, позволяющие устанавливать соединение с рабочим столом, для пользователя имеются в открытом доступе. Что, если существуют такие приложения,
  о которых пользователь не имеет никакого представления? Получается, данное программное обеспечение будет являться потенциальной угрозой, так как
  с помощью него можно подключиться к удаленному рабочему столу без разрешения самого пользователя.
  
  Таким образом, в данной ситуации необходимо проанализировать возможную угрозу, используя различные методы обнаружения RDP-сессии.
  
  \subsection{Клавиатурный мониторинг}

  Одним из таких методов обнаружения является клавиатурный мониторинг. Допустим может возникнуть ситуация, когда злоумышленник уже смог подключиться к
  удаленному рабочему столу локального пользователя. Его сертификат оказался действительным, и он может полностью взять под свой контроль чужой компьютер.
  И тогда у локального пользователя остается мало шансов предотвратить утечку информации.
  
  И здесь одним из способов обнаружения RDP-сессии может быть полезен кейлоггер. Это программное обеспечение или аппаратное устройство, регистрирующее
  различные действия пользователя — нажатия клавиш на клавиатуре компьютера, движения и нажатия клавиш мыши и т.д. \cite{keylog}. В данном случае кейлоггер
  выполняет следующие задачи:
  
  \begin{itemize}
    \item регистрирует нажатия клавиш на клавиатуре и мыши компьютера;
    \item при достижении заданного предела символов отправляет по протоколу SMTP (Simple Mail Transfer Protocol --- протокол передачи почты) сообщение на
    электронную почту;
    \item создает log-файл, в который делаются записи названия клавиш и их время удержания, а также названия клавиш мыши и их позиция на экране, представленная
    в виде двух координат.  
  \end{itemize}

  Чтобы обезопасить свой компьютер от сторонних подключений, локальный пользователь запускает программу кейлоггер, написанную на языке Python. Из рисунка \ref{cmd}
  видно, что программа запрашивает логин и пароль от электронной почты, на которую будут отправляться сообщения.
  
  \begin{figure}[H]
    \centering
    \includegraphics[width=0.9\textwidth]{photo/cmd.png}
    \caption{Окно программы при работе кейлоггера}
    \label{cmd}
  \end{figure}

  Допустим злоумышленник уже знает IP-адрес и имя пользователя компьютера, и ему удалось подключится к удаленному рабочему столу. На рисунке \ref{input1} показано
  подключение к удаленному рабочему столу, а именно к компьютеру с IP-адресом 192.168.10.229. Стоит отметить, что здесь подключение производятся через
  известное приложение для удаленного рабочего стола устройства Windows. Конечно, злоумышленник будет использовать программу, сигнатура которой никому другому
  неизвестна. Поэтому программа <<Подключение к удаленному рабочему столу>> используется в качестве примера.

  На рисунке \ref{input1} показан ввод различной информации при успешно установленной RDP-сессии.

  \begin{figure}[H]
    \centering
    \includegraphics[width=0.9\textwidth]{photo/input1.png}
    \caption{Демонстрация работы с удаленным рабочим столом}
    \label{input1}
  \end{figure}


  Допустим злоумышленник решил попытаться зайти на сайт ipsilon.sgu.ru, зная логин и пароль некоторого пользователя, как показано на рисунке \ref{input2}.

  \begin{figure}[H]
    \centering
    \includegraphics[width=0.9\textwidth]{photo/input2.png}
    \caption{Ввод логина и пароля, осуществляемый на сайте ipsilon.sgu.ru}
    \label{input2}
  \end{figure}

  И после набора определенного количества символов осуществляется отправка письма на электронную почту. На рисунке \ref{mailcheck} видно,
  что письмо доставлено на введенную локальным пользователем почту.

  \begin{figure}[H]
    \centering
    \includegraphics[width=0.9\textwidth]{photo/mail.png}
    \caption{Получение электронного письма}
    \label{mailcheck}
  \end{figure}

  На рисунке \ref{mail1} показано содержимое сообщения. Записи << <l> >> и << <r> >> обозначаются нажатия левой и правой кнопки мыши соответственно.

  \begin{figure}[H]
    \centering
    \includegraphics[width=0.9\textwidth]{photo/alsomail.png}
    \caption{Содержимое электронного письма}
    \label{mail1}
  \end{figure}

  Получив письмо, пользователь может начать действовать, пытаясь изолировать злоумышленника от интернета или просто зайдя под своей учетной записью на своем
  устройстве. Таким образом RDP-сессия будет преддотвращена. При закрытии консоли или нажатии комбинации клавиш $Ctrl + f5$, программа завершает 
  свою работу и в итоге получается log-файл, показанный на рисунке \ref{r1}. 
  
  
  \begin{figure}[H]
    \centering
    \includegraphics[width=0.4\textwidth]{photo/log-file.png}
    \caption{Содержимое файла file.log}
    \label{r1}
  \end{figure}

  Возможно ситуация, в которой злоумышленнику удается получить контроль над удаленным рабочим столом, маловероятна. Однако это не исключено. Ведь RDP блокирует
  текущую сессию, где компьютер в данный момент работает, не позволяя двум пользователям видеть действия другого. В таком случае локальному пользователю будет сложно
  остановить утечку данных, так как он не знает, когда была установлена RDP-сессия. Поэтому представленный в работе кейлоггер здесь может оказаться полезным.

  \subsection{Другие варианты обнаружения RDP-сессии}

  В Windows существуют логи RDP подключений, которые позволяют администраторам терминальных RDS серверов получить информацию о том, какие пользователи подключались
  к серверу, когда сеанс был начат или завершен. Информация об этих событиях содержится в журналах Windows. Ее можно просмотреть, открыв коллекцию средств администрирования
  --- <<Управление компьютером>>, позволяющая управлять локальным или удаленным компьютером. Открыв <<Просмотр событий>> необходимо рассмотреть следующее:

  \begin{itemize}
    \item Network Connection ---  установка сетевого подключения к серверу от RDP клиента пользователя;
    \item Authentication --- успешная или неуспешная аутентификация пользователя на сервере;
    \item Session Disconnect/Reconnect --- события отключения/переподключения сессии имеют разные коды в зависимости от того, что вызвало отключение
    пользователя (отключение по неактивности, выбор пункта Disconnect в сессии, завершение RDP сессии другим пользователем или администратором и т.д.);
    \item Logon и Logoff --- RDP вход в систему и выход из системы. 
  \end{itemize}

  В перечисленных событиях необходимо анализировать значения EventID. Благодаря им, можно узнать некоторую информацию о возможных подключениях.
  В основном просмотр логов RDP подключений осуществляется уже после того, как произошла какая-либо утечка информации. Благодаря просмотру событий
  пользователь может предотвратить будущие атаки злоумышленников, однако он уже не сможет вернуть похищенные данные, которые производились в результате
  установки посторонних RDP-сессий.

  \conclusion

  На основании проведенной работы можно предположить, что RDP далеко не самый защищенный протокол. Хотя корпорация Microsoft регулярно выпускает обновления
  для своего программного обеспечения. Однако RDP-сессия становится уязвимой из-за упущений в безопасности, например, из-за некорректной конфигурации сервисов или
  установки устаревших обновлений системы. В таком случае злоумышленник может использовать такие просчеты в своих целях. Конечно, далеко не за всем можно уследить
  и не всегда получается предвидеть возможную угрозу. Но лучше попробовать предположить возможное решение данной проблемы, вместо того чтобы вообще о ней не думать. 
  

  \begin{thebibliography}{15}
    \bibitem{1}
    Книга Ибе О.С. «Компьютерные сети и службы удаленного доступа» / пер. с англ. -
    Москва, издательство: «ДМК Пресс», Яз. рус.
    \bibitem{2}
    Удалённый рабочий стол RDP: как включить и как подключиться по RDP [Электронный ресурс] / URL:https://hackware.ru/?p=11835 (дата обращения 03.05.2022), Яз. рус.
    \bibitem{userdp1}
    How to use remote desktop [Электронный ресурс] / URL: https://support.microsoft.com/en-us/windows/how-to-use-remote-desktop-5fe128d5-8fb1-7a23-3b8a-41e636865e8c (дата обращения 27.05.2022), Яз. англ.
    \bibitem{userdp2}
    Статья <<Как исправить ошибку удаленного рабочего стола не удается подключиться к удаленному компьютеру>> [Электронный ресурс] / URL: https://okdk.ru/kak-ispravit-oshibku-udalennogo-rabochego-stola-ne-udaetsya-podkljuchitsya-k-udalennomu-kompjuteru/ 
    (дата обращения 27.05.2022), Яз. рус.
    \bibitem{rdp2}
    Документация Remote Utilities <<RDP>> [Электронный ресурс] / URL:  https://www.remoteutilities.com/support/docs/rdp/ (дата обращения 27.05.2022), Яз. англ.
    % \bibitem{3}
    % Документация по устранению неполадок служб удаленных рабочих стола для Windows Server [Электронный ресурс] / URL: https://inlnk.ru/bvOV0 (дата обращения 04.05. 2022), Яз. рус.
    \bibitem{keylog}
    Статья в википедии <<Кейлоггер>>[Электронный ресурс] / URL: https://ru.wikipedia.org/wiki/Кейлогер (дата обращения 28.05.2022), Яз. рус.
    \bibitem{rdpudp}
    Документация Microsoft <<Протоколы>> [Электронный ресурс] / URL: https://docs.microsoft.com/en-us/openspecs/windows_protocols/ms-rdpeudp2/d8bf9a56-90f3-4608-8f98-9600ed69876b (дата обращения 28.05.2022), Яз. рус.
    \bibitem{rdp1}
    Статья <<Wireshark Tutorial: Decrypting RDP Traffic>> [Электронный ресурс] / URL: https://unit42-paloaltonetworks-com.translate.goog/wireshark-tutorial-decrypting-rdp-traffic/?_x_tr_sl=en\&_x_tr_tl=ru\&_x_tr_hl=ru\&_x_tr_pto=op,wapp
    (дата обращения 28.05.2022), Яз. англ.

  \end{thebibliography}

  \appendix

    \section{Код keylogger-win.py}
    \inputminted[fontsize=\footnotesize]{Python}{code/keylogger-win.py}

    % \section{Код keylogger-linux.py}
    % \inputminted[fontsize=\footnotesize]{text}{keylogger-linux.py}

    % \section{Код analyze-data.py}
    % \inputminted[fontsize=\footnotesize]{Python}{code/analyze-data.py}

\end{document}